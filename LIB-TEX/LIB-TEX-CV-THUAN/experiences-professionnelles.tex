
%*********************************************************************************************
%
%*********************************************************************************************
\section{Expériences professionnelles}


%==========================================================================================
%
%==========================================================================================
% \vspace{-10pt}
%\subsection{Recherche et Développement}

	\cventry
	{06/2016-présent}
	{Ingénieur de Recherche}
	{CEA NanoInnov, Labo LCE}{Saclay, Paris}{France}{}
	\cvlistitem{Projet: Dynamique instrumentation et Analyse des codes binaires}
	\cvlistitem{Environnements: Linux(Ubuntu, Fedora, Debian arm), C/C++, MPI, OpenMP, PAPI, TAU, PIN-Tool Intel, DynamoRIO, DynInst-API, Valgrind, gprof, git, eclipse, Make, Shellscript, QtCreator, VisJS, json}
	
	%\vspace{-5pt}
% 	\cventry
% 	{01/2016-05/2016}
% 	{Enseignant-chercheur en Informatique(associé, télétravail)}
% 	{Faculté des technologies de l'information et de la communication}
% 	{Université de Cantho}
% 	{Vietnam}
% 	{}

	%-------------------------------------------------------------------
	%
	%-------------------------------------------------------------------
	\vspace{-5pt}
	\cventry
	{03/2013-12/2015}
	{Ingénieur de Développement}
	{Société Adanam-Technology(société innovante)}{Paris}{France}{}
	\cvlistitem{Projet: Confidentiel des données sur les clouds, Compression de données}
% 	\cvlistitem{Environnements: Linux, C/C++, OpenMP, MongoDB, TAU, Valgrind, gprof, git, Java, Python(script), 
% 				API Java pour Clouds (Amazon AWS S3, 
% 				Amazon AWS RDS, Google Drive, DropBox, Box.net), 
% 				eclipse, ant, Make, Bash Shell}
	\cvlistitem{ Environnements: 
					Linux(Ubuntu, Fedora), 
					Java(POO, Spring, Swing, JavaFX, JNI),
					Java Webservice Jersey,
					MongoDB, C/C++, Valgrind, git,  
					API Java pour Clouds (	Amazon AWS S3, 
											Amazon AWS RDS, 
											Google Drive, 
											DropBox, 
											Box.net, 
											OneDrive, 
											Hubic, 
											OpenStack), 
					API Java(Facebook, Twitter),
					Apache, eclipse, maven, ant, make, gprof,
					NodeJS (études)
				}
				
	%\cvlistitem{Études: NodeJS}
	%\cvlistitem{Math et Algo: Distance levenstein, algorithmes de compression}


	%-------------------------------------------------------------------
	%
	%-------------------------------------------------------------------
	%\vspace{-5pt}
	\cventry
	{09/2012-03/2013}
	{Ingénieur de Recherche et Développement, Post-doc}
	{Projet: Calcul Haute Performance dans la Géophysique, Entreprise BRGM}{}{}{}
	\cvlistitem{Optimisation de codes ondes3d pour la simulation d'ondes sismique en utilisant la méthode différence-finie}
	\cvlistitem{Environnements: Cluster, MPI(GNU, Intel), OpenMP, C/C++, Bash Shell, Python(script)}
	\cvlistitem{Outils: Eclipse, Gestion de jobs (OAR, QSUB ), Make, TAU, Intel Vtune, Valgrind, gprof}


	%-------------------------------------------------------------------
	%
	%-------------------------------------------------------------------
	\vspace{-7pt}
	\cventry{02/2008-12/2011}
	{Doctorant}
	{Projet eXtenGIS, Partenaires: Société Géo-Hyd, Labo. LIFO, ISTO-Tours)}
	{}
	{}
	{}
	\cvlistitem{Construction d'une Plateforme de calculs répartis sur cluster: Calcul des bassins versants, des flux d'accumulation, extraction des réseaux hydrographiques dans un gros modèle numérique de terrain}
	\cvlistitem{Environnements: Cluster Linux, MPI, MPI-IO(PVFS), OpenMP, C/C++, shellscript, cvs, svn, make, gnuplot, valgrind, gprof}

	
	%-------------------------------------------------------------------
	%
	%-------------------------------------------------------------------
	%\vspace{-7pt}
	\cventry
	{09/2005-12/2007}
	{Enseignant-chercheur en Informatique}
	{Faculté des technologies de l'information et de la communication}
	{Université de Cantho}
	{Vietnam}
	{}
% 	\cvlistitem{}
% 	\cvlistitem{Environnements: Courbe Bézier, OpenGL, C/C++}

	%-------------------------------------------------------------------
	%
	%-------------------------------------------------------------------
	\vspace{-7pt}
	\cventry
	{02-06/2005}
	{Stage de Master M2 Recherche}
	{Équipe SIRV, Labo IRIT }
	{Toulouse}
	{France}
	{}
	\cvlistitem{Simulation comportementale de la circulation à moto dans les grandes villes vietnamiennes}
	\cvlistitem{Environnements: Courbe Bézier, OpenGL, C/C++}

	

	%-------------------------------------------------------------------
	%
	%-------------------------------------------------------------------
	\vspace{-7pt}
	\cventry
	{10/2003-08/2004}
	{Enseignant-chercheur en Informatique}
	{Faculté des technologies de l'information et de la communication}
	{Université de Cantho}
	{Vietnam}
	{}	
	
	%-------------------------------------------------------------------
	%
	%-------------------------------------------------------------------
	\vspace{-7pt}
	\cventry
	{03-09/2003}
	{Stage de D.E.S.S I.G.S.I}
	{Centre ressources du TICE, Université Toulouse I}
	{Toulouse}{France}{}
	\cvlistitem{Construction des animations des cours en Informatique pour Formation ouverte et à distance}
	\cvlistitem{Environnements: Flash, ActionScript, Php, HTML, Javascript}

	

	%-------------------------------------------------------------------
	%
	%-------------------------------------------------------------------
	\vspace{-7pt}
	\cventry
	{10/2001-04/2002}
	{Mémoire de fin d'études en Ingénieur}
	{Université de Cantho, Vietnam}
	{}
	{}
	{}
	\cvlistitem{Système de garde et de désignation de tâche(Application au SGBD géographique GBASE)}
	\cvlistitem{Environnements: Visual C++ (MFC, Socket, Client/Serveur, TCP/IP), XML, Sécurité, SIG, SQL}



	%-------------------------------------------------------------------
	%
	%-------------------------------------------------------------------
% 	\vspace{-7pt}
% 	\cventry
% 	{02-03/2002}
% 	{Stage d'entreprise}
% 	{Université de Cantho, Vietnam}
% 	{}
% 	{}
% 	{}
% 	\cvlistitem{``Programmation d'application Web pour la Gestion''}
% 	\cvlistitem{Environnements: SQL Server, ASP, HTML, Merise}



	
	
% %*********************************************************************************
% %
% %*********************************************************************************
% % \vspace{-10pt}
% \subsection{Activités d'enseignement}
% 
% 
% 	%-------------------------------------------------------------------
% 	%
% 	%-------------------------------------------------------------------
% 	\vspace{-3pt}
% 	\cventry
% 	{09/2010--09/2012}
% 	{A.T.E.R (Temps partiel)}
% 	{UFR Sciences}
% 	{Université d'Orléans}
% 	{France}
% 	{}
% 	\cvlistdoubleitem[]{Programmation en Java (Jeux Réversi, Pentago)}{Architecture des ordinateurs (TD, TP)}
% 	\cvlistdoubleitem[]{Calcul Intensif(Cours, MPI-1, MPI-2)}{Arbre Couvrant de poids minimum}
% 
% 
% 	
% 	%-------------------------------------------------------------------
% 	%
% 	%-------------------------------------------------------------------
% 	\cventry
% 	{2002--2008}
% 	{Enseignant-chercheur en Informatique}
% 	{}
% 	{Université de Cantho}
% 	{Vietnam}
% 	{}
% 	\cvlistitem{Programmation d'Applications réseau en Java(POO, Socket, Thread, Client/Server), RPC/RMI, JDBC}
% 	\cvlistitem{Programmation Orienté d'Objet en C++/Java}
% 	\cvlistitem{Programmation d'Applications Web(Php, MySQL, Javascript, HTML, HTTP)}
% 	\cvlistitem{Linux et Logiciels Libres(Bash, administration de base du système(Apache, DNS, NFS, FTP), Shell)}
% 	\cvlistitem{Programmation en Java - Programme de l'Aptech(POO, GUI Swing, Socket, Thread, Client/Server), JDBC}
	
	

% %***************************************************************************************************
% %
% %***************************************************************************************************
% % \vspace{-10pt}
% \subsection{Présentation des activités de recherche}
% 
% 
% 	%-------------------------------------------------------------------
% 	%
% 	%-------------------------------------------------------------------
% 	\cventry
% 	{06/2011}
% 	{Présentation en anglais}
% 	{Titre: Parallel Computing Flow Accumultion in large digital elevation models 
% 		\newline{} "The International Conference on Computational Science", ICCS 2011
% 	}
% 	{}
% 	{Singapour}
% 	{}
% 
% 	
% 	%-------------------------------------------------------------------
% 	%
% 	%-------------------------------------------------------------------
% 	\cventry
% 	{13/12/2011}
% 	{Soutenance de thèse en français}
% 	{Titre: 
% Extensibilité des moyens de traitements pour les données issues des vastes systèmes d'informations géographiques 
% 		\newline{} LIFO Bâtiment 3IA, Université d'Orléans,
% 	}
% 	{Orléans}
% 	{France}
% 	{}
% 	
% 	
% 	%-------------------------------------------------------------------
% 	%
% 	%-------------------------------------------------------------------
% 	\cventry
% 	{02/11/2009}
% 	{Présentation en français}
% 	{Titre: "Calcul parallèle des bassins versants dans de grands modèles numérique de terrain", Journée des Doctorants}
% 	{LIFO - Université d'Orléans}
% 	{Orléans, France}
% 	{}
% 	
% 	
% 	%-------------------------------------------------------------------
% 	%
% 	%-------------------------------------------------------------------
% 	\cventry
% 	{08/2009}
% 	{Présentation en anglais}
% 	{
% 		Titre : ``Parallel computing of catchment basins in large digital elevation model'' 
% 		\newline{} Conférence HPCA'09 "The 2$^{nd}$ International Conference on High Performance Computing and Application"
% 	}
% 	{Université de Shanghai}
% 	{Shanghai, Chine}
% 	{}


	%-------------------------------------------------------------------
	%
	%-------------------------------------------------------------------

	
	
	
	
% \section{Experience}
% \subsection{Vocational}
% \cventry{year--year}{Job title}{Employer}{City}{}{General description no longer than 1--2 lines.\newline{}%
% Detailed achievements:%
% \begin{itemize}%
% \item Achievement 1;
% \item Achievement 2, with sub-achievements:
%   \begin{itemize}%
%   \item Sub-achievement (a);
%   \item Sub-achievement (b), with sub-sub-achievements (don't do this!);
%     \begin{itemize}
%     \item Sub-sub-achievement i;
%     \item Sub-sub-achievement ii;
%     \item Sub-sub-achievement iii;
%     \end{itemize}
%   \item Sub-achievement (c);
%   \end{itemize}
% \item Achievement 3.
% \end{itemize}}
% \cventry{year--year}{Job title}{Employer}{City}{}{Description line 1\newline{}Description line 2}
% \subsection{Miscellaneous}
% \cventry{year--year}{Job title}{Employer}{City}{}{Description}

